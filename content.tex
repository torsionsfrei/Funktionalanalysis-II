\begin{theo}[Continuous Functinoal Calculus]
Let $T\in\mathcal L(\mathfrak H)$ be self-adjoint. There is a unique $\Phi\colon C(\sigma(T))\to \mathcal L(\mathfrak H)$ st. 
\begin{enumerate}
    \item $\Phi(z) = T, \Phi(1)=\Id$
    \item (a) $\Phi$ is linear, (b) $\Phi$ is multiplicative (c) $\Phi(\overline f)= \Phi(f)^*$
    \item $\Phi$ is continuous.
\end{enumerate}
\end{theo}

\begin{proof}
Uniqueness: By multiplicativity, we hvae $\Phi(z^n) = T^n$, hence, by the linearity of $\Phi$, it must be unique on the space of all polynomials. Furthermore $\sigma(T)$ is compact in $\mathbb R$ and polynomials are dnese in $\mathcal C(\sigma(T))$. Due to the continuit of $\Phi$, it is unique on $\mathcal C((\sigma(T))$. 
\end{proof}